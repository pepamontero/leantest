\documentclass{article}

% Paquetes usados
\usepackage{graphicx} % Imágenes
\usepackage{amsmath} % Algunos símbolos matemáticos
\usepackage{amsfonts}
\usepackage{array} % Matrices, tablas
\usepackage{xcolor} % Colores de texto

% Distancias entre párrafos, quitar sangrías
\setlength{\parindent}{0pt}
\setlength{\parskip}{.8em}

% Espaciados entre palabras en el justificado.
\sloppy

% Título
\title{Formalización de las matemáticas con Lean. Un caso de estudio: Resultados de Topología General.}
\author{Pepa Montero Jimena}
\date{}


\begin{document}

\maketitle

\section{Introducción}

\section{Lean Theorem Prover}

El proyecto Lean fue iniciado en 2012 por Leonardo de Moura, que formada parte de Microsoft Research Redmond. Es un proyecto activo con una visión a largo plazo, y con miras en el potencial de automatización en las demostraciones matemáticas.\cite{avigad2015theorem}

Lean es un lenguaje de programación funcional y un asistente de demostración. Un asistente de demostración es taltaltal. \cite{leanprover2024} Por otro lado, al tratarse de un lenguaje de programación, permite taltaltal (cita)

En particular, Lean 4 es una reprogramación de Lean 3 (explicar) \cite{moura2021lean}, donde Lean 3 era taltal y decir que está implementado en C?

Explicar lo que es mathlib


\subsection{Teoría de tipos dependientes}

Explicar como funciona Lean con teoría de tipos.

Ejemplos de tipos en Lean para entender como funciona la movida??


\subsection{Lean como asistente de demostración}

Nos centramos en la función de theorem proving.

Import mathlib

Explicar la sintaxis de como se expresa un resultado. Ejemplos de proposiciones algunas más simples y otras más complejas? Ej. modus ponens vs. algún otro resultado.

Hay varias formas de demostrar cosas en Lean. Notar que en este proyecto nos vamos a centrar en pruebas en la forma de tácticas.


\subsection{Modo táctico}

Dado un enunciado, un término de tipo prueba es una representación de una demostración matemática de la veracidad de este. Las tácticas son comandos, o instrucciones, que describen cómo se construye esta prueba.

Explicar lo que son las hipótesis y los goals

Cada vez que utilicemos una táctica, las hipótesis y los objetivos que aparecen en el Lean Info se actualizarán.

Para construir una prueba en el modo táctico, escribimos :=

Explicar lo que es el Lean Infoview.

[ejemplo]

En general, y sobretodo en los casos sencillos, existe una relación bastante clara entre estos comandos (tácticas) y las expresiones que utilizamos en las demostraciones formales a las que estamos acostumbrados.

A continuación vamos a ver una a una las tácticas que más se utilizan en las demostraciones.

\subsection{Tácticas}

\subsubsection{Tácticas básicas}

\begin{itemize}
    \item \textbf{\textcolor{blue}{intro}}
\end{itemize}
    
Imaginemos que queremos demostrar un resultado del estilo de "si A, entonces B". Normalmente, empezaríamos la demostración diciendo "Supongamos que se da A. Veamos que entonces se tiene B"

En el modo táctico, utilizamos la táctica \textbf{\textcolor{blue}{intro}} con esta finalidad. Si nuestro objetivo es "$\vdash$ A $\rightarrow$ B", donde A y B son elementos de tipo Prop ("A B : Prop"), entonces "\textbf{\textcolor{blue}{intro}} hA" añade la hipótesis "hA : A", y nuestro objetivo pasa a ser "$\vdash$ B".

%[EJEMPLO intro_1]

Otra forma de utilizar "intro" es en las expresiones de la forma "Para todo $x \in X$, $x$ tiene la propiedad $P$". Normalmente, empezaríamos diciendo "sea un $x \in X$ fijo pero arbitrario... veamos que $x$ satisface $P$".

En Lean, utilizamos "intro" con este fin. Si tenemos como objetivo "$\vdash \forall$ x : X, P x", donde P es una función que asigna a cada x en X un término de tipo Prop, entonces "intro x" añade la hipótesis "x : X", y el objetivo pasa a ser "$\vdash$ P x".

%[EJEMPLO intro_2]

\subsubsection{sorry}

Notemos que en ambos ejemplos anteriores, la demostración termina con "sorry". "sorry" es una táctica también, que indica a Lean que la demostración no está terminada, pero que no queremos terminarla por ahora. Si ponemos el cursor sobre la palabra "example", que está subrayada en amarillo, obtenemos el mensaje "declaration uses sorry" (es decir, la prueba no está completa). Podemos utilizar resultados incompletos en otras pruebas, pero estas pruebas también estarán marcadas como incompletas.

\subsubsection{exact}

Otra táctica fundamental es "exact". Usamos "exact" cuando el objetivo es exactamente ([poner esto como una explicación al pie de página]o definicionalmente, es decir es igual por debajo para Lean) igual a una de nuestras hipótesis. Por ejemplo:

% Ejemeplos exact

En el segundo ejemplo, para Lean es exactamente lo mismo "$x \in A \cap B$" que "$x \in A \land x \in B$".

\subsubsection{rfl}

Utilizamos "rfl" cuando tenemos en el objetivo una expresión con una igualdad, una doble implicación o cualquier otra relación de equivalencia, en la que los términos a ambos lados son exactamente (o definicionalmente) iguales.

% Ejemplos rfl

\subsubsection{trivial}

La táctica "trivial" se utiliza cuando el objetivo es "True" o definicionalmente igual a "True". En algún momento de todo esto tengo que utilizar la palabra unificar? Supongo jajaja

% ejemplos trivial

En el último ejemplo, "x $\in$ Set.univ" está definido en Lean como True para todo x.

\subsubsection{apply}

La táctica "apply" consiste en aplicar Modus Ponens. Si quiero demostrar B, y tengo como hipótesis "A implica B", entonces basta con demostrar A. Bueno esto hay que explicarlo mejor.

Notar que también sirve para expresiones que aunque no sean implicaciones, son definicionalmente iguales a implicaciones. Por ejemplo, en Lean 4, $A \subset B$ es definicionalmente igual a $x \in A \implies x \in B$. Por tanto podemos hacer lo siguiente:

% ejemplos apply

También podemos usar apply en otra hipótesis (en vez de en el objetivo!). Para ello utilizamos "apply h1 at h2".

% ejemplo apply at

\subsubsection{by\_contra}

Esta táctica es equivalente a utilizar el método de reducción al absurdo. Se puede utilizar en cualquier momento, y lo que hace es añadir una nueva hipótesis resultante de negar el objetivo, y el objetivo pasa a ser "False".

Para llegar a demostrar False, necesitaremos tener dos hipótesis que sean contradictorias (lo que normalmente conocemos como "llegar a una contradicción"), utilizaremos entonces "exact h1 h2", donde h1 siempre debe ser la negación natural de h2 (y no al revés).

% ejemplos by_contra

Notar que si una de mis hipótesis es False (o definicionalmente igual a False), entonces puedo demostrar cualquier cosa utilizando by\_contra. (último ejemplo)

\subsubsection{Resumen de tácticas básicas}

% TABLA CON LAS TÁCTICAS
% poner bien el ancho de las filas :/

\renewcommand{\arraystretch}{2}

\begin{center}
\begin{tabular}{|  m{8em}  |m{8em} |m{8em}  |} 
  \hline
  \textbf{antes} & \textbf{táctica} & \textbf{después} \\
  \hline
  $\vdash$ P $\rightarrow$ Q & intro hP & \parbox{8em}{hP : P \\ $\vdash$ Q} \\ 
  \hline
  $\vdash \forall$x : X, P x & intro x &  \parbox{8em}{x : X \\ $\vdash$ P x} \\ 
  \hline
  \parbox{8em}{h : P \\ $\vdash$ P}& exact h & No goals\\ 
  \hline
  $\vdash$ x = x & rfl & No goals\\ 
  \hline
  $\vdash$ x $\leftrightarrow$ x & rfl & No goals\\ 
  \hline
  $\vdash$ True & trivial & No goals \\ 
  \hline
  \parbox{8em}{h : P $\rightarrow$ Q \\ $\vdash$ Q} & apply h & $\vdash$ P\\ 
  \hline
  \parbox{8em}{h1 : P $\rightarrow$ Q \\ h2 : P} & apply h1 at h2 & h2 : Q\\ 
  \hline
  $\vdash$ P & by\_contra h & \parbox{8em}{h : $\neg$ P \\ $\vdash$ False} \\\hline
\end{tabular}
\end{center}

Pasamos a ver otras tácticas más avanzadas.

\subsubsection{Simplificar expresiones}

rw y simp, dsimp

\subsection{Trabajar con varias tesis a la vez}

En ocasiones tendremos varias tesis en vez de sólo una (en la siguiente sección podemos ver varios ejemplos en los que esto ocurre). En este caso, podremos ver arriba del InfoView el número de tesis activas que tenemos, y a continuación veremos la primera tesis, seguida de la siguiente.

Para poder trabajar sólo con una tesis a la vez (menos congestión visual), podemos separar las tesis utilizando puntos : · (alt+3). Ejemplo: (poner imagen de como se ve el InfoView)

\subsubsection{Trabajar con conectores lógicos}

En muchas ocasiones estaremos trabajando con los conectores $\land, \lor, \leftrightarrow$. Podemos utilizar varias tácticas para facilitar esta tarea.

\begin{itemize}
    \item \textbf{left, right}
\end{itemize}

Para tesis de la forma "A ó B" basta demostrar que se da A (análogamente, basta demostrar que se da B). Para esto, utilizamos la táctica "left" (respectivamente "right"). Es decir, le comunico a Lean que me voy a centrar sólo en la parte izquierda (resp. derecha) del "ó".

% ejemplos left, right

Notar que, en el último ejemplo, como x in A union B es definicionalmente equivalente a x in A or x in B, podemos utilizar right directamente.

Además, en hipótesis de la forma A $\land$ B, podemos querer utilizar sólo A (ó sólo B). Podemos utilizar h.left (h.right), respectivamente.

\begin{itemize}
    \item \textbf{cases'}
\end{itemize}

La táctica "cases'" (pie de página: existe una táctica cases pero es menos potente que esta, y esta es la que utilizaremos normalmente) se utiliza sobre una hipótesis de la forma "P $\lor$ Q", y queremos proceder de forma distinta en caso de que P y de que Q.

Es el equivalente a dividir una demostración en casos en el lenguaje natural. Movida: esto genera varios goals. Esto habría que explicarlo bien. Poner que gestionamos los distintos goals con · ?

Esta táctica también tiene otro uso práctico interesante para $\land$ : si se aplica a una hipótesis de la forma "P $\land$ Q", dividimos esta hipótesis en dos. Esto nos puede servir en ocasiones para simplificar las hipótesis y trabajar más fácilmente con ellas.

\begin{itemize}
    \item \textbf{constructor}
\end{itemize}

Podemos utilizar costructor cuando tengamos tesis que se puedan separar en dos partes, por ejemplo tesis de la forma "P $\land$ Q" se convierten en dos tesis distintas, por un lado P y por otro Q. Igualmente, tesis de la forma "P $\leftrightarrow$ Q" se divide en "P $\rightarrow$ Q" y "Q $\rightarrow$ P".


\begin{center}
\begin{tabular}{ | m{8em} | m{8em}| m{8em} | } 
  \hline
  \textbf{antes} & \textbf{táctica} & \textbf{después} \\
  \hline
  $\vdash$ P $\lor$ Q & left & $\vdash$ P \\
  \hline
  $\vdash$ P $\lor$ Q & right & $\vdash$ Q \\
  \hline
  \parbox{8em}{h : P $\lor$ Q \\ $\vdash$ goal} & cases' h with hP hQ & \parbox{8em}{goal 1: \\ hP : P \\ $\vdash$ goal \\[1ex] goal 2: \\ $~~$ hQ : Q \\ $~~$ $\vdash$ goal} \\
  \hline
  h : P $\land$ Q & cases' h with hP hQ & \parbox{8em}{hP : P \\ hQ : Q }\\
  \hline
  
\end{tabular}
\end{center}

% Bibliografía

\bibliographystyle{plain}
\bibliography{references.bib}

\end{document}
